%
% This is a borrowed LaTeX template file for lecture notes for CS267,
% Applications of Parallel Computing, UCBerkeley EECS Department.
% Now being used for CMU's 10725 Fall 2012 Optimization course
% taught by Geoff Gordon and Ryan Tibshirani.  When preparing 
% LaTeX notes for this class, please use this template.
%
% To familiarize yourself with this template, the body contains
% some examples of its use.  Look them over.  Then you can
% run LaTeX on this file.  After you have LaTeXed this file then
% you can look over the result either by printing it out with
% dvips or using xdvi. "pdflatex template.tex" should also work.
%

\documentclass[twoside]{article}
\setlength{\oddsidemargin}{0.25 in}
\setlength{\evensidemargin}{-0.25 in}
\setlength{\topmargin}{-0.6 in}
\setlength{\textwidth}{6.5 in}
\setlength{\textheight}{8.5 in}
\setlength{\headsep}{0.75 in}
\setlength{\parindent}{0 in}
\setlength{\parskip}{0.1 in}

%
% ADD PACKAGES here:
%

\usepackage{amsmath,amsfonts,graphicx}
\usepackage[utf8]{inputenc}

%
% The following commands set up the lecnum (lecture number)
% counter and make various numbering schemes work relative
% to the lecture number.
%
\newcounter{lecnum}
\renewcommand{\thepage}{\thelecnum-\arabic{page}}
\renewcommand{\thesection}{\thelecnum.\arabic{section}}
\renewcommand{\theequation}{\thelecnum.\arabic{equation}}
\renewcommand{\thefigure}{\thelecnum.\arabic{figure}}
\renewcommand{\thetable}{\thelecnum.\arabic{table}}

%
% The following macro is used to generate the header.
%
\newcommand{\lecture}[4]{
   \pagestyle{myheadings}
   \thispagestyle{plain}
   \newpage
   \setcounter{lecnum}{#1}
   \setcounter{page}{1}
   \noindent
   \begin{center}
   \framebox{
      \vbox{\vspace{2mm}
    \hbox to 6.28in { {\bf Web Information Retrieval
	\hfill Spring 2018} }
       \vspace{4mm}
       \hbox to 6.28in { {\Large \hfill #2  \hfill} }
       \vspace{2mm}
       \hbox to 6.28in { { \textbf{Members}: \textit{#3} \hfill  #4} }
      \vspace{2mm}}
   }
   \end{center}
   \markboth{Lecture #1: #2}{Lecture #1: #2}

   %{\bf Note}: {\it LaTeX template courtesy of UC Berkeley EECS dept.}

   \vspace*{4mm}
}
%
% Convention for citations is authors' initials followed by the year.
% For example, to cite a paper by Leighton and Maggs you would type
% \cite{LM89}, and to cite a paper by Strassen you would type \cite{S69}.
% (To avoid bibliography problems, for now we redefine the \cite command.)
% Also commands that create a suitable format for the reference list.
\renewcommand{\cite}[1]{[#1]}
\def\beginrefs{\begin{list}%
        {[\arabic{equation}]}{\usecounter{equation}
         \setlength{\leftmargin}{2.0truecm}\setlength{\labelsep}{0.4truecm}%
         \setlength{\labelwidth}{1.6truecm}}}
\def\endrefs{\end{list}}
\def\bibentry#1{\item[\hbox{[#1]}]}

%Use this command for a figure; it puts a figure in wherever you want it.
%usage: \fig{ NUMBER}{SPACE-IN-INCHES}{CAPTION}
\newcommand{\fig}[1]{
			\begin{center}
			#1
			\end{center}
	}
% Use these for theorems, lemmas, proofs, etc.
\newtheorem{theorem}{Theorem}[lecnum]
\newtheorem{lemma}[theorem]{Lemma}
\newtheorem{proposition}[theorem]{Proposition}
\newtheorem{claim}[theorem]{Claim}
\newtheorem{corollary}[theorem]{Corollary}
\newtheorem{definition}[theorem]{Definition}
\newenvironment{proof}{{\bf Proof:}}{\hfill\rule{2mm}{2mm}}

% **** IF YOU WANT TO DEFINE ADDITIONAL MACROS FOR YOURSELF, PUT THEM HERE:

\newcommand\E{\mathbb{E}}

\begin{document}
\lecture{1}{WIR Project Proposal}{Angelo Catalani, Valerio Colitta, Alessandro Lo Presti}

\section{Proposal}
The paper describes a new approach to compute similarity between documents and queries. Not only does it take into account tf-idf, but also SIMILARITY among terms via a new formula.\\
Terms are identified through WordNet, where they are linked to each other in different ways. Given a particular synset (sense) for a term, you can traverse the whole network to find related terms (superclasses, subclasses, etc).\\
The paper introduces a specific metric for quantifying the similarity of two words by measuring shortest path between their senses in the WordNet graph.\\
We implement a new metric, using the WordNet graph. In particular we establish that the similarity depends on the least common ancestor in the term-hypernym path (the least the better).

\section{Outline}
\begin{itemize}
\item \textbf{VSM} le caratteristiche e le limitazioni con la similarity.\\
\item \textbf{GVSM}, introduzione di un nuovo modello vettoriale e di una estensione della vecchia cosine similarity che tiene conto della relazione semantica dei termini.\\
\item \textbf{WORDNET}, e il fatto che il paper lo usi per trovare la similarity in base a delle metriche specificate ad-hoc : \texttt{SCM}, \texttt{SPE}, \texttt{SR}.\\
\item \textbf{NOI}, che creiamo una nuova similarity measure basata su wordnet, che usa il least commont ancestor.\\
\item \textbf{COMPARAZIONE}, Termine-Termine usando tre dataset. Document-Query usando NPL. Quest'ultimo, usando le metriche precision and recall graph.

\end{itemize}




\end{document}





